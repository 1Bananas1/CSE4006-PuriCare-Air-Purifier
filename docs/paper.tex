\documentclass[conference]{IEEEtran}
\IEEEoverridecommandlockouts
% The preceding line is only needed to identify funding in the first footnote. If that is unneeded, please comment it out.
\usepackage{cite}
\usepackage{amsmath,amssymb,amsfonts}
\usepackage{algorithmic}
\usepackage{graphicx}
\usepackage{comment}
\usepackage[T1]{fontenc}
\usepackage{textcomp}
\usepackage{fixltx2e}
\usepackage{url}
\usepackage{seqsplit}
\usepackage{xcolor}
\usepackage{enumitem}
\setlist[enumerate,1]{label=\Alph*., leftmargin=25pt, itemsep=8pt}
\setlist[enumerate,2]{label=\arabic*), leftmargin=20pt, itemsep=3pt}
\setlist[itemize]{label=\textbullet, leftmargin=20pt, itemsep=2pt}
\def\BibTeX{{\rm B\kern-.05em{\sc i\kern-.025em b}\kern-.08em
    T\kern-.1667em\lower.7ex\hbox{E}\kern-.125emX}}
\begin{document}

\title{PureCare Air Purifier\\

\thanks{}
}

\author{
\IEEEauthorblockN{Jimmy MacDonald}
\IEEEauthorblockA{\textit{dept. of Computer Science} \\
\textit{Hanyang University}\\
Seoul, South Korea \\
james.macdonald.1@slu.edu}
\and
\IEEEauthorblockN{Yim Soobeen}
\IEEEauthorblockA{\textit{dept. of Information System}\\
\textit{Hanyang University}\\
Seoul, South Korea \\
s00been@hanyang.ac.kr}
\and
\IEEEauthorblockN{Kim Minjin}
\IEEEauthorblockA{\textit{dept. of Information System} \\
\textit{Hanyang University}\\
Seoul, South Korea \\
idid02@hanyang.ac.kr}
\and
\IEEEauthorblockN{Kim Yeonwoo}
\IEEEauthorblockA{\textit{dept. of Computer Science} \\
\textit{Hanyang University}\\
Seoul, South Korea \\
bbongvong@hanyang.ac.kr}
}

\maketitle

\begin{abstract}
    Air purifiers have become more and more common in people's houses, but as houses gain more technology, it seems that air purifiers haven't had any major improvements. We propose a smart air purifier which learns and adapts how it runs based on it's users. Specifically, by monitoring and recognizing when users are sick, gaining insights about air quality outside as well as seasonal allergies, and smart scheduling on when to run the air purifier to minimize electricity usage.
    We are going to implement 3 models-cough detection, sniff detection, machine control. Respiratory diseases are air-sensitive, which means complicated factors, such as air quality-include dust, viruses, temperature, and humidity affect users. Our expanded goal is to implement total air care control. 
\end{abstract}

\begin{IEEEkeywords}
    Smart Air Purifier, AQI
\end{IEEEkeywords}

\begin{table}[htbp]
\caption{Role Assignments}
\begin{center}
\small
\setlength{\tabcolsep}{6pt} 
\renewcommand{\arraystretch}{1.3} 
\begin{tabular}{|p{2cm}|p{1.5cm}|p{4cm}|}
\hline
\multicolumn{1}{|c|}{\textbf{Roles}} & \multicolumn{1}{c|}{\textbf{Name}} & \multicolumn{1}{c|}{\textbf{Task description and etc.}} \\ \hline
\centering User & \centering Yim Soobin & Identifies user needs and improves product usability. Thinks of ideas, designs functions, and suggests practical ways to implement features that address real-world problems. \\ \hline
\centering Customer & \centering Jimmy MacDonald & Conducts competitive analysis and plans differentiation strategy. Examines competitor services, identifies market gaps, and designs UI elements that attract customers. \\ \hline
\centering Software developer & \centering Kim Yeonwoo & Converts requirements into functional software. Writes code, implements features, debugs programs, monitors performance, and ensures all components execute properly. \\ \hline
\centering Development Manager & \centering Kim Minjin & Manages overall development process and team coordination. Creates plans, assigns roles, monitors progress, manages schedules, and ensures smooth communication among team members. \\ \hline
\end{tabular}%
\label{tab1}
\end{center}
\end{table}

\newpage

\section{Introduction}
    \subsection{Motivation}
        As the world continues to industrialize, air quality has declined as a direct consequence. As harmful particulate matter such as PM\textsubscript{2.5}, PM\textsubscript{10}, and other volatile organic matter enter the atmosphere, residents find difficulty in breathing under these conditions. The World Health Organization (WHO), estimates that 99\% of the global population breathes air that exceeds guidelines, leading to millions of premature deaths annually from various diseases \cite{b1}. 

        Despite living in a technology-driven age, air purifiers have seemed to lack any significant advances, remaining more reactive than proactive. Modern air purifiers activate only once particles have been detected in the air, leaving users exposed to a harmful environment until the device responds. While this reactive approach provides some benefit, the delay between exposure and activation can lead to numerous audible health symptoms, such as sneezing or coughing, potentially interrupting important moments such as presentations or meeting. Furthermore, current air purifiers require human configuration and continue to run continuously regardless if anyone is home resulting in a waste of electricity.

        We propose PureCare, a truly intelligent air purification system that acts as a proactive health assistant rather than a passive appliance. PureCare distinguishes itself through several key innovations: Audible Health Symptom Recognition technology that detects sneezing, coughing, and snoring to automatically adjust purification settings; geofencing capabilities that optimize operation based on occupancy; API integration with local air quality data to prepare for poor outdoor conditions; personalized care recommendations that learn individual sensitivities over time; and sleep mode optimization for nighttime respiratory issues. Unlike traditional purifiers that simply filter air, PureCare anticipates needs, provides gentle human-like notifications, and creates safer environments before problems arise.
\newpage
    \subsection{Similar Products}
        \begin{enumerate}
            \item LG PuriCare Objet Collection AI+ 360\textdegree\ Air Purifier
                \begin{itemize}
                    \item Detects harmful gas and fumes from three major sick-building substances-formaldehyde ammonia, and volatile oraganinc compounds (VOCs)-and automatically purifies the air according to gas type and pollution level.
                    \item Detects ammonia, the main source of pet waste odors.
                    \item Detects contaminants such as cooking fumes.
                    \item "AI Customized Operation" learns and analyzes indoor air quality every hour. Using accumulated data, it identifies times when the air is clean, stops the purifier's fan, and dims the display to save energy.
                    \item By autonomously analyzing air quality and adjusting operating intensity, it can reduce power consumption by up to 50 percent or more compared with the conventional AI mode, helping lower electricity costs.
                \end{itemize}
            \item SAMSUNG BESPOKE Cube Air Infinite Line
                \begin{itemize}
                    \item Provides an AI-powered integrated personalized cleaning solution.
                    \item "Customized Clean AI+" compares and learns indoor and outdoor air quality, and pre-cleans when deterioration is predicted.
                    \item "AI Saving Mode" automatically adjusts airflow or stops the fan when indoor air improves, cutting energy use by up to 45% on the 100㎡ model.
                    \item "Customized Clean AI+" is certified with "AI+ Certification" by the Korea Standards Association.
                \end{itemize}
            \item SK Wellness Robot
                \begin{itemize}
                    \item The Wellness Robot by SK Namoothix's brand, NAMUHX, removes harmful substances that enter during ventilation to maintain air quality. It also offers performance equivalent to six air purifiers. All of its air solution functions are controlled entirely by voice without touch, and through the AI control system, device monitoring and remote support for maintenance are provided, greatly enhancing user convenience.
                \end{itemize}
        \end{enumerate}
        
\section{Requirement Analysis}
    \subsection{\textbf{App Installation and First Launch}}
        \subsubsection{FR-001: Application and First Launch}

            \begin{enumerate}
                \item Platform Availability
                    \begin{enumerate}
                        \item iOS Platform: Available on App Store for iOS 12.0+ (iPhone 6S and newer)
                        \item Android Platform: Available on Google Play Store for Android 10+ (Armv7 and ARM64 architectures)
                    \end{enumerate}
                \item First Launch Experience
                    \begin{enumerate}
                        \item Splash Screen: Displays application logo and version number for 1-2 seconds
                        \item Permission Requests: Camera (for QR scanning), Notifications (for device alerts), and optional Location (for weather recommendations)
                    \end{enumerate}
            \end{enumerate}

    \subsection{\textbf{Login/Sign Up}}
        \subsubsection{FR-002: User Authentication}
            \begin{enumerate}
                \item Welcome Screen\\
                    Displays after splash screen with application logo, tagline, and two buttons: "Login" and "Sign Up"
                \item Login Screen\\
                    Contains email and password fields, "Remember me" checkbox, "Forgot Password?" link, and "Don't have an Account? Sign Up" link
                \item Sign-up Flow
                    \begin{enumerate}
                        \item Page 1: Basic account information (Full Name, Email, Password, Confirm Password, Terms Agreement)
                        \item Page 2: Optional profile information (Phone, Profile Picture, DOB, Country)
                    \end{enumerate}
            \end{enumerate}
    \subsection{\textbf{Product Setup via QR Code}}
        \subsubsection{FR-003: Product Registration}
            \textit{Note: Only accessible after successful login/sign up}
            \begin{enumerate}
                \item Access Points
                    \begin{enumerate}
                        \item Automatically shown to new users after login (if no devices registered)
                        \item "Add Device" button on Home Page
                        \item "+" button in navigation bar
                    \end{enumerate}
                \item Initial Screen for New Users\\
                If user has zero registered devices, shows empty state with title "Connect your first device", description "Scan the QR code on your device to get started", "Scan QR Code" button, and "Enter serial number manually" link
                \item QR Code Scanning Process
                    \begin{enumerate}
                        \item Camera Activation: Checks auth token validity. If invalid, redirects to login with message "Please login to register devices"
                        \item QR Detection and Validation: Scans QR code, validates against database, checks if device already registered. Error shown: "This device is registered to another account"
                        \item Device Linking: Links device to user account, stores User ID, Device Serial Number, Device Model, and Registration Timestamp
                    \end{enumerate}
                \item Product Information Confirmation\\
                    Displays product image, model name, serial number, and user's name with prompt "Register this device to \{Name\}'s account?" with confirm/cancel buttons
                \item Registration Complete\\
                    Shows success message "Device registered successfully", sends email notification, offers "Go home" or "Add another device" options, then navigates to Home Page
            \end{enumerate}

        \subsection{\textbf{Home Page}}
            \subsubsection{FR-004: Home Page}
                \textit{Note: Only accessible after successful login/sign up}
                    \begin{enumerate}
                        \item Home Page Access\\
                            Displayed after successful login, device registration (first-time), or tapping "home" tab. Checks authentication on load—expired tokens redirect to Login
                        \item Home Page Layout
                            \begin{enumerate}
                                \item Header: User greeting "Welcome back \{user\}", current AQI conditions, notification bell, settings gear
                                \item Device List: Shows all registered devices or "No devices yet..." if empty. Each card displays Device Name, Model, Status, Last Active time, and Quick Action button. Includes "add device" floating action button
                            \end{enumerate}
                        \item Device Status Synchronization\\
                            Loads device list from database, queries IoT server status, updates UI with real-time data, auto-refreshes every 30 seconds
                        \item Navigation Bar\\
                            The Home Page must include the bottom navigation bar as specified in FR-005, with the "Home" tab highlighted/active
                    \end{enumerate}

        \subsection{\textbf{Navigation Bar}}
            \subsubsection{FR-005: Navigation Bar}
                \begin{enumerate}
                    \item Bottom Navigation Structure\\
                    Persistent across all main screens with three tabs arranged from left to right: Automation (clock icon), Home (house icon), Settings (Gear icon). When logged out, tapping any tab redirects to login
                \end{enumerate}
        \subsection{\textbf{Automation Page}}
            \subsubsection{FR-006: Automation}
                \textit{Note: Only accessible after successful login/sign up}
                    \begin{enumerate}
                        \item Routine Screen Sections
                            \begin{enumerate}
                                \item Screen Header
                                    \begin{enumerate}
                                        \item \textit{FR-RTS-1.0}: The screen must display a static title "Routines"
                                        \item \textit{FR-RTS-1.1}: Below the title must display a static helper text "Automate your Air Purifier by building Routines. For each Routine, schedule events with a start time and action for your selected devices."
                                        \item \textit{FR-RTS-1.2}: The header must contain an "Add" icon button (a + symbol) in the top-right corner.
                                    \end{enumerate}
                                \item Routine List
                                    \begin{enumerate}
                                        \item \textit{FR-RTS-2.0}: The system must fetch and display the user's routines
                                        \item \textit{FR-RTS-2.1}: If the list of routines exceeds the vertical viewport, the list must be vertically scrollable.
                                        \item \textit{FR-RTS-2.2}: Each routine in the list must be displayed as a distinct "Routine Card" (see FR-RTS-3.0).
                                    \end{enumerate}
                                \item Routine Card
                                    \begin{enumerate}
                                        \item \textit{FR-RTS-3.0}: Each Routine Card must display the user-defined name of the routine (e.g., "My Routine").
                                        \item \textit{FR-RTS-3.1}: The card must contain a visual summary of the routine's logic.
                                        \item \textit{FR-RTS-3.2}: The visual summary must display a chronological list of triggers (clock icons) aligned vertically.
                                        \item \textit{FR-RTS-3.3}: The visual summary must display a corresponding list of device actions (e.g., Air Purifier, Thermostat) to the right of the triggers.
                                        \item \textit{FR-RTS-3.4}: A vertical connector line must visually link the triggers and actions, with nodes indicating each time/action pair.
                                        \item \textit{FR-RTS-3.5}: Device icons in the summary must visually represent their target state:
                                        \begin{enumerate}
                                            \item On/Active: Icon is illuminated (e.g., glowing, in color).
                                            \item Off/Inactive: Icon is dim (e.g., greyed out).
                                        \end{enumerate}
                                    \end{enumerate} 
                                \item Navigation
                                    \begin{enumerate}
                                        \item \textit{FR-ITS-4.0}: Tapping the "Add" icon (FR-RTS-1.2) must navigate the user to the "Create Routine" screen.
                                        \item \textit{FR-ITS-4.1}: Tapping on any existing "Routine Card" (FR-RTS-3.0) must navigate the user to the "Edit Routine" screen, pre-populated with that routine's data.
                                        \item \textit{FR-ITS-4.2}: The screen must display a "New Routine" card-style button below the list of existing routines.
                                        \item \textit{FR-ITS-4.3}: The "New Routine" button must display the text "New Routine" and an "Add" icon (+).
                                        \item \textit{FR-ITS-4.4}: Tapping the "New Routine" button (FR-RTS-4.2) must navigate the user to the "Create Routine" screen (same action as FR-RTS-4.0).
                                    \end{enumerate}
                                \item Navigation Bar
                                    \begin{enumerate}
                                        \item \textit{FR-RTS-5.0}: The Automation Page must include the bottom navigation bar as specified in FR-005, with the "Automation" tab highlighted/active
                                    \end{enumerate}
                            \end{enumerate}
                        \end{enumerate}

        \subsection{\textbf{Add Routine Screen}}
            \subsubsection{FR-007: Add Routine Screen}
                \begin{enumerate}
                    \item Screen Header
                        \begin{enumerate}
                            \item \textit{FR-RTA-1.0}: The screen must display a title "Custom Routine" centered at the top
                            \item \textit{FR-RTA-1.1}: The header must contain a "Back" icon button (← arrow) in the top-left corner. When clicked, if the user has unsaved changes, a dialog must warn the user about unsaved changes with options to "Discard" or "Keep Editing"
                            \item \textit{FR-RTA-1.2}: The header must contain a "Delete" icon button (trash bin icon) in the top-right corner. When clicked, a confirmation dialog must appear with "Delete Routine?" message and "Cancel"/"Delete" options
                        \end{enumerate}
                    \item Routine Name Section
                        \begin{enumerate}
                            \item \textit{FR-RTA-2.0}: Below the header, the screen must display an editable routine name field with default text "[Time] Routine" (e.g., "10:49 AM Routine")
                            \item \textit{FR-RTA-2.1}: An edit icon (pencil) must appear next to the routine name to indicate editability
                            \item \textit{FR-RTA-2.2}: Tapping the routine name or edit icon must open an inline text input or modal for editing
                            \item \textit{FR-RTA-2.3}: A toggle switch must appear on the right side of the routine name to enable/disable the entire routine
                            \item \textit{FR-RTA-2.4}: The toggle switch must be green when enabled and gray when disabled
                        \end{enumerate}
                    \item Day Selection Section
                        \begin{enumerate}
                            \item \textit{FR-RTA-3.0}: A "When" label must be displayed above the day selection buttons
                            \item \textit{FR-RTA-3.1}: Seven day buttons must be displayed in a horizontal row: S, M, T, W, TH, F, SA
                            \item \textit{FR-RTA-3.2}: Each day button must be a rounded square with uniform size and spacing
                            \item \textit{FR-RTA-3.3}: Selected days must have white background; unselected days must have dark/transparent background
                            \item \textit{FR-RTA-3.4}: Multiple days can be selected simultaneously
                            \item \textit{FR-RTA-3.5}: At least one day must be selected for the routine to be valid
                            \item \textit{FR-RTA-3.6}: Tapping a selected day must deselect it; tapping an unselected day must select it
                        \end{enumerate}
                    \item Events Section
                        \begin{enumerate}
                            \item \textit{FR-RTA-4.0}: An "Events" label must be displayed as a section header
                            \item \textit{FR-RTA-4.1}: A copy/duplicate icon must appear in the top-right corner of the Events section to duplicate the entire routine
                            \item \textit{FR-RTA-4.2}: The section must display a scrollable list of event cards
                            \item \textit{FR-RTA-4.3}: Each event must be contained in a distinct card with rounded corners and dark background
                        \end{enumerate}
                    \item Event Card Structure
                        \begin{enumerate}
                            \item \textit{FR-RTA-5.0}: Each event card must display a colored gradient icon on the left (unique color per event)
                            \item \textit{FR-RTA-5.1}: The icon must contain a clock symbol to indicate time-based triggering
                            \item \textit{FR-RTA-5.2}: Event cards must display the event name (e.g., "Untitled Event 65011")
                            \item \textit{FR-RTA-5.3}: The second line must show the action state (ON/OFF), time in 24-hour format, and action type (e.g., "Set Scene")
                            \item \textit{FR-RTA-5.4}: Format must be: "[STATE], [TIME], [ACTION]" (e.g., "ON, 18:00, Set Scene")
                            \item \textit{FR-RTA-5.5}: A chevron (>) must appear on the right side of each card to indicate it's tappable
                            \item \textit{FR-RTA-5.6}: Tapping an event card must navigate to the event detail/edit screen
                            \item \textit{FR-RTA-5.7}: Event cards must support swipe gestures for quick delete actions
                        \end{enumerate}
                    \item New Event Button
                        \begin{enumerate}
                            \item \textit{FR-RTA-6.0}: A "New Event" button must appear at the bottom of the events list
                            \item \textit{FR-RTA-6.1}: The button must have a "+" icon on the right side
                            \item \textit{FR-RTA-6.2}: Tapping "New Event" must navigate to the event creation screen
                            \item \textit{FR-RTA-6.3}: The button must maintain consistent styling with other UI elements (dark background, light text)
                        \end{enumerate}
                    \item Event Ordering and Management
                        \begin{enumerate}
                            \item \textit{FR-RTA-7.0}: Events must be automatically sorted chronologically by their trigger time
                            \item \textit{FR-RTA-7.1}: Users must be able to reorder events via drag-and-drop (long-press and drag)
                            \item \textit{FR-RTA-7.2}: A maximum of 20 events per routine must be enforced
                            \item \textit{FR-RTA-7.3}: If a user attempts to add more than 20 events, an error message must display: "Maximum 20 events per routine"
                        \end{enumerate}
                    \item Save and Validation
                        \begin{enumerate}
                            \item \textit{FR-RTA-8.0}: Changes must auto-save after 2 seconds of inactivity
                            \item \textit{FR-RTA-8.1}: A routine is valid only if: at least one day is selected and at least one event exists
                            \item \textit{FR-RTA-8.2}: If validation fails, the back button must show the unsaved changes warning
                            \item \textit{FR-RTA-8.3}: Upon successful save, a toast notification must display: "Routine saved"
                        \end{enumerate}
                \end{enumerate}
        \subsection{\textbf{Control Page}}
            \subsubsection{FR-008: Device Control Page}
                \begin{enumerate}
                    \item Access Control
                        \begin{enumerate}
                            \item \textit{FR-CTL-1.0}: Accessible only after successful device pairing (FR-003)
                            \item \textit{FR-CTL-1.1}: Must verify device connectivity before displaying controls
                            \item \textit{FR-CTL-1.2}: If device is offline, display "Device Offline" message with retry button
                        \end{enumerate}
                    \item Power Control
                        \begin{enumerate}
                            \item \textit{FR-CTL-2.0}: Display prominent ON/OFF toggle button at the top of control interface
                            \item \textit{FR-CTL-2.1}: Toggle must show current device state with visual feedback (colored indicator)
                            \item \textit{FR-CTL-2.2}: Power state changes must be reflected on device within 2 seconds
                            \item \textit{FR-CTL-2.3}: When device is OFF, all other controls must be visually disabled/grayed out
                        \end{enumerate}
                    \item Fan Speed Control
                        \begin{enumerate}
                            \item \textit{FR-CTL-3.0}: Display horizontal percentage slider for fan speed control (0-100\%)
                            \item \textit{FR-CTL-3.1}: Slider must show current fan speed percentage as text above or beside slider
                            \item \textit{FR-CTL-3.2}: Fan speed adjustments must be sent to device in real-time during slider movement
                            \item \textit{FR-CTL-3.3}: Slider must be disabled when device is in Timer Mode or Auto Mode
                        \end{enumerate} 
                    \item Timer Mode
                        \begin{enumerate}
                            \item \textit{FR-CTL-4.0}: Provide timer mode selection with options: OFF, 4hr, 6hr, 8hr
                            \item \textit{FR-CTL-4.1}: Display timer as segmented control or dropdown menu
                            \item \textit{FR-CTL-4.2}: When timer is active, display countdown showing remaining time
                            \item \textit{FR-CTL-4.3}: Device must automatically turn OFF when timer expires
                            \item \textit{FR-CTL-4.4}: User must receive push notification when timer completes: "Air purifier timer finished"
                            \item \textit{FR-CTL-4.5}: Timer mode overrides Auto Mode when activated
                        \end{enumerate}            
                    \item Child Lock
                        \begin{enumerate}
                            \item \textit{FR-CTL-5.0}: Display Child Lock toggle switch with lock icon
                            \item \textit{FR-CTL-5.1}: When enabled, all physical controls on device must be disabled
                            \item \textit{FR-CTL-5.2}: Child Lock state must persist until manually disabled via app
                            \item \textit{FR-CTL-5.3}: Lock status must be visually indicated on device (LED indicator or display message)
                            \item \textit{FR-CTL-5.4}: App controls remain functional regardless of Child Lock state
                        \end{enumerate}
                    \item Operation Modes
                        \begin{enumerate}
                            \item \textit{FR-CTL-6.0}: Provide mode selection between "Manual Mode" and "Auto Mode"
                            \item \textit{FR-CTL-6.1}: Display current active mode prominently with visual distinction
                            \item \textit{FR-CTL-6.2}: Mode changes must be confirmed with user before applying
                        \end{enumerate}
                
                    \item Manual Mode Controls
                        \begin{enumerate}
                            \item \textit{FR-CTL-7.0}: In Manual Mode, enable direct fan speed control via percentage slider
                            \item \textit{FR-CTL-7.1}: Manual controls must override any automatic adjustments
                            \item \textit{FR-CTL-7.2}: Display "Manual Mode" indicator when active
                            \item \textit{FR-CTL-7.3}: User must manually adjust all settings (fan speed, timer, etc.)
                            \item \textit{FR-CTL-7.4}: Manual Mode remains active until user switches to Auto Mode or device power cycle
                        \end{enumerate}                
                    \item Auto Mode Operation
                        \begin{enumerate}
                            \item \textit{FR-CTL-8.0}: In Auto Mode, device must take sensor readings every 15 minutes
                            \item \textit{FR-CTL-8.1}: Fan speed must automatically adjust based on environmental sensor data
                            \item \textit{FR-CTL-8.2}: Display "Auto Mode" indicator and show "Next sensor reading in: X minutes"
                            \item \textit{FR-CTL-8.3}: Auto Mode adjustments must consider: air quality, temperature, humidity, and detected health symptoms
                            \item \textit{FR-CTL-8.4}: Fan speed slider must be disabled/read-only in Auto Mode, showing current automatic setting
                            \item \textit{FR-CTL-8.5}: User must be able to view sensor reading history and auto-adjustment reasoning
                            \item \textit{FR-CTL-8.6}: Auto Mode must respect Child Lock and Timer Mode settings when active
                        \end{enumerate}
                    \item Control Interface Layout
                        \begin{enumerate}
                            \item \textit{FR-CTL-9.0}: All controls must be organized in logical sections with clear visual separation
                            \item \textit{FR-CTL-9.1}: Emergency/priority controls (Power, Child Lock) must be easily accessible
                            \item \textit{FR-CTL-9.2}: Current device status must be displayed at top: power state, mode, timer status
                            \item \textit{FR-CTL-9.3}: Interface must update in real-time to reflect device state changes
                            \item \textit{FR-CTL-9.4}: Loading states must be shown during control command transmission
                        \end{enumerate}
                \end{enumerate}
        \subsection{\textbf{Settings Page}}
            \subsubsection{FR-009: Settings Page (Main Menu)}
                \textit{Note: Only accessible after successful login/sign up}
                \begin{enumerate}
                    \item Settings Page Access
                        \begin{enumerate}
                            \item \textit{FR-SET-1.0}: The Settings Page is accessed by tapping the "Settings" tab in the bottom navigation bar (FR-005)
                            \item \textit{FR-SET-1.1}: The page must check authentication on load—expired tokens redirect to Login
                            \item \textit{FR-SET-1.2}: The Settings Page must include the bottom navigation bar with the "Settings" tab highlighted/active
                        \end{enumerate}
                    \item Screen Header
                        \begin{enumerate}
                            \item \textit{FR-SET-2.0}: The screen must display "Settings" as the page title at the top-left
                            \item \textit{FR-SET-2.1}: An optional icon (such as a palette or settings gear) may appear in the top-right corner
                        \end{enumerate}
                    \item Settings Menu List
                        \begin{enumerate}
                            \item \textit{FR-SET-3.0}: The screen must display a vertically scrollable list of settings options
                            \item \textit{FR-SET-3.1}: Each menu item must have an icon on the left, title text in the center, and a chevron (>) on the right
                            \item \textit{FR-SET-3.2}: The following options must be displayed in order:
                                \begin{itemize}
                                    \item Account (with user/profile icon)
                                    \item My Devices (with device icon)
                                    \item Location (with location/pin icon)
                                    \item Privacy (with shield/lock icon)
                                \end{itemize}
                            \item \textit{FR-SET-3.3}: Tapping any menu item must navigate to its respective detail screen
                            \item \textit{FR-SET-3.4}: Menu items must have sufficient touch target size (minimum 44pt height)
                        \end{enumerate}
                \end{enumerate}
            \subsubsection{FR-010: Account Screen}
                \begin{enumerate}
                    \item Screen Structure
                        \begin{enumerate}
                            \item \textit{FR-ACC-1.0}: Display "Account" or user's account name as the title
                            \item \textit{FR-ACC-1.1}: Include a back button (<) in the top-left corner to return to Settings menu
                            \item \textit{FR-ACC-1.2}: Display a section header "Account Settings"
                        \end{enumerate}
                    \item Account Options
                        \begin{enumerate}
                            \item \textit{FR-ACC-2.0}: Display "Edit Account" option with a chevron (>)
                            \item \textit{FR-ACC-2.1}: Tapping "Edit Account" navigates to account editing screen
                            \item \textit{FR-ACC-2.2}: Display "Log Out" option without a chevron
                            \item \textit{FR-ACC-2.3}: Tapping "Log Out" must show confirmation dialog: "Are you sure you want to log out?"
                            \item \textit{FR-ACC-2.4}: Upon logout confirmation, clear authentication token and redirect to Login screen
                        \end{enumerate}
                    \item Edit Account Screen
                        \begin{enumerate}
                            \item \textit{FR-ACC-3.0}: The Edit Account screen must allow users to edit their account name/display name
                            \item \textit{FR-ACC-3.1}: Display current account name in an editable text field
                            \item \textit{FR-ACC-3.2}: Include a "Save" button to save changes
                            \item \textit{FR-ACC-3.3}: Include a "Cancel" button or back button to discard changes
                            \item \textit{FR-ACC-3.4}: Show success message "Account updated" after successful save
                            \item \textit{FR-ACC-3.5}: Validate that account name is not empty and is between 2-50 characters
                        \end{enumerate}
                \end{enumerate}
            \subsubsection{FR-011: My Devices Screen}
                \begin{enumerate}
                    \item Screen Structure
                        \begin{enumerate}
                            \item \textit{FR-DEV-1.0}: Display "My Devices" as the screen title
                            \item \textit{FR-DEV-1.1}: Include a back button (<) in the top-left corner
                            \item \textit{FR-DEV-1.2}: Include a "+" button in the top-right corner to add new devices
                            \item \textit{FR-DEV-1.3}: Tapping the "+" button must trigger the device registration flow (FR-003)
                        \end{enumerate}
                    \item Device List Display
                        \begin{enumerate}
                            \item \textit{FR-DEV-2.0}: Display all registered devices in a scrollable vertical list
                            \item \textit{FR-DEV-2.1}: If no devices are registered, show empty state: "No devices registered" with "Add Device" button
                            \item \textit{FR-DEV-2.2}: Each device must be displayed in a card with the following information:
                                \begin{itemize}
                                    \item Device icon (left side)
                                    \item Device name with optional status indicator icon
                                    \item Room/location name (below device name)
                                    \item Device version or model number (below location)
                                    \item Three-dot menu button (...) on the right side
                                \end{itemize}
                            \item \textit{FR-DEV-2.3}: Device cards must have rounded corners and appropriate padding/spacing
                            \item \textit{FR-DEV-2.4}: Cards should maintain consistent styling with other UI elements (dark background for dark mode)
                        \end{enumerate}
                    \item Device Options Menu
                        \begin{enumerate}
                            \item \textit{FR-DEV-3.0}: Tapping the three-dot (...) button must display a context menu/modal overlay
                            \item \textit{FR-DEV-3.1}: The menu must appear as a popup/modal centered or near the device card
                            \item \textit{FR-DEV-3.2}: The menu must have a light background with rounded corners (iOS-style)
                            \item \textit{FR-DEV-3.3}: The menu must include the following options in order:
                                \begin{itemize}
                                    \item "Rename" (with edit/pencil icon)
                                    \item "Identify" (with location/search icon)
                                    \item "Settings" (with gear icon)
                                    \item "Delete" (with trash/minus icon, displayed in red/warning color)
                                \end{itemize}
                            \item \textit{FR-DEV-3.4}: Tapping outside the menu must close it without taking action
                            \item \textit{FR-DEV-3.5}: Each menu option must have an icon on the right side
                        \end{enumerate}
                    \item Rename Device
                        \begin{enumerate}
                            \item \textit{FR-DEV-4.0}: Tapping "Rename" must open a dialog/modal with a text input field
                            \item \textit{FR-DEV-4.1}: The dialog must display current device name as default value
                            \item \textit{FR-DEV-4.2}: Include "Cancel" and "Save" buttons
                            \item \textit{FR-DEV-4.3}: Device name must be 1-50 characters
                            \item \textit{FR-DEV-4.4}: Show success message "Device renamed" after save
                        \end{enumerate}
                    \item Identify Device
                        \begin{enumerate}
                            \item \textit{FR-DEV-5.0}: Tapping "Identify" must send a command to the physical device to identify itself
                            \item \textit{FR-DEV-5.1}: The device should flash its LED, beep, or otherwise indicate which device it is
                            \item \textit{FR-DEV-5.2}: Show message "Identifying device..." while command is being sent
                            \item \textit{FR-DEV-5.3}: Show success message "Device should now be blinking/beeping"
                        \end{enumerate}
                    \item Device Settings
                        \begin{enumerate}
                            \item \textit{FR-DEV-6.0}: Tapping "Settings" must navigate to device-specific settings screen
                            \item \textit{FR-DEV-6.1}: Device settings may include: firmware version, WiFi connection, update options, etc.
                        \end{enumerate}
                    \item Delete Device
                        \begin{enumerate}
                            \item \textit{FR-DEV-7.0}: Tapping "Delete" must show confirmation dialog: "Remove [Device Name]? This device will be removed from your account."
                            \item \textit{FR-DEV-7.1}: Dialog must have "Cancel" and "Remove" buttons
                            \item \textit{FR-DEV-7.2}: "Remove" button must be in red/warning color
                            \item \textit{FR-DEV-7.3}: Upon confirmation, remove device from user account
                            \item \textit{FR-DEV-7.4}: Show success message "Device removed"
                            \item \textit{FR-DEV-7.5}: Update the device list to remove the deleted device
                        \end{enumerate}
                \end{enumerate}
            \subsubsection{FR-012: Location Screen}
                \begin{enumerate}
                    \item Screen Structure
                        \begin{enumerate}
                            \item \textit{FR-LOC-1.0}: Display "Location" as the screen title (centered)
                            \item \textit{FR-LOC-1.1}: Include a back button (<) in the top-left corner
                            \item \textit{FR-LOC-1.2}: Include a "Save" button in the top-right corner
                        \end{enumerate}
                    \item Location Search
                        \begin{enumerate}
                            \item \textit{FR-LOC-2.0}: Display a search bar below the header with placeholder text "Search"
                            \item \textit{FR-LOC-2.1}: Include a magnifying glass icon on the left side of the search bar
                            \item \textit{FR-LOC-2.2}: Include a location/GPS icon button on the right side of the search bar
                            \item \textit{FR-LOC-2.3}: Tapping the GPS icon must request device location and auto-populate nearest city
                            \item \textit{FR-LOC-2.4}: As user types, show autocomplete suggestions for cities/locations
                            \item \textit{FR-LOC-2.5}: User can select a location from search results or autocomplete suggestions
                        \end{enumerate}
                    \item Location Selection
                        \begin{enumerate}
                            \item \textit{FR-LOC-3.0}: Selected location must be saved to the user's profile in the database
                            \item \textit{FR-LOC-3.1}: Tapping "Save" must save the location and return to Settings menu
                            \item \textit{FR-LOC-3.2}: Show success message "Location saved"
                            \item \textit{FR-LOC-3.3}: If user taps back button without saving, show warning: "Discard changes?"
                            \item \textit{FR-LOC-3.4}: Location data should include: city name, country, latitude, longitude
                        \end{enumerate}
                    \item Location Usage
                        \begin{enumerate}
                            \item \textit{FR-LOC-4.0}: Saved location will be used for external AQI (Air Quality Index) API integration
                            \item \textit{FR-LOC-4.1}: Location will be used for weather-based automation recommendations
                        \end{enumerate}
                \end{enumerate}
            \subsubsection{FR-013: Privacy Screen}
                \begin{enumerate}
                    \item Screen Structure
                        \begin{enumerate}
                            \item \textit{FR-PRIV-1.0}: Display "Privacy" as the screen title
                            \item \textit{FR-PRIV-1.1}: Include a back button (<) in the top-left corner
                        \end{enumerate}
                    \item Privacy Content
                        \begin{enumerate}
                            \item \textit{FR-PRIV-2.0}: Display privacy policy statement or summary
                            \item \textit{FR-PRIV-2.1}: Include link to full privacy policy document: "Read Full Privacy Policy"
                            \item \textit{FR-PRIV-2.2}: Optionally include toggles for:
                                \begin{itemize}
                                    \item Data collection consent
                                    \item Audio processing (for cough/sneeze detection)
                                    \item Location services
                                \end{itemize}
                            \item \textit{FR-PRIV-2.3}: Include explanation text for each privacy setting
                            \item \textit{FR-PRIV-2.4}: Include "Download My Data" option
                            \item \textit{FR-PRIV-2.5}: Include "Delete My Account" option at bottom in red text
                        \end{enumerate}
                \end{enumerate}
            \subsubsection{FR-014: Empathetic Intelligence Notifications}
                \begin{enumerate}
                    \item Notification Permissions
                        \begin{enumerate}
                            \item \textit{FR-NOTIF-1.0}: During app setup (FR-001), request push notification permissions with explanation: "Allow notifications to receive caring health insights and device updates"
                            \item \textit{FR-NOTIF-1.1}: If permissions denied initially, provide in-app prompt in Settings to enable notifications
                            \item \textit{FR-NOTIF-1.2}: User must be able to customize notification types and frequency in Settings menu
                            \item \textit{FR-NOTIF-1.3}: Notifications must respect device "Do Not Disturb" and quiet hours settings
                        \end{enumerate}
                    \item Notification Design Principles
                        \begin{enumerate}
                            \item \textit{FR-NOTIF-2.0}: All notifications must use empathetic, human-like language that demonstrates care and understanding
                            \item \textit{FR-NOTIF-2.1}: Notifications must be actionable, providing clear next steps or automatic solutions
                            \item \textit{FR-NOTIF-2.2}: Tone must be supportive and non-alarming, avoiding medical terminology or urgent language
                            \item \textit{FR-NOTIF-2.3}: Messages must personalize to individual users when multiple users detected
                            \item \textit{FR-NOTIF-2.4}: Include gentle emoji or icons where appropriate to enhance emotional connection
                        \end{enumerate}
                    \item Responsive Notifications
                        \begin{enumerate}
                            \item \textit{FR-NOTIF-3.0}: Monitor cough/sneeze detection events and trigger notifications when threshold exceeded
                            \item \textit{FR-NOTIF-3.1}: Threshold definition: 5+ cough events or 8+ sneeze events within 30-minute window
                            \item \textit{FR-NOTIF-3.2}: Example responsive notification: "We've noticed you have been coughing quite a bit lately, I've turned on Pollen Protection Mode to help out "
                            \item \textit{FR-NOTIF-3.3}: Automatic action must be taken simultaneously with notification (e.g., activate protection mode)
                            \item \textit{FR-NOTIF-3.4}: Follow-up notification after 2 hours: "How are you feeling? The air should be much cleaner now"
                            \item \textit{FR-NOTIF-3.5}: Avoid sending duplicate responsive notifications within 4-hour window for same symptom type
                        \end{enumerate}
                    \item Predictive Notifications
                        \begin{enumerate}
                            \item \textit{FR-NOTIF-4.0}: Integrate external AQI data with user health sensitivity profiles to predict issues
                            \item \textit{FR-NOTIF-4.1}: Learn individual user sensitivities over time (Person A reacts to pollen, Person B to dust, etc.)
                            \item \textit{FR-NOTIF-4.2}: Multi-user household example: "Pollen levels in the area are rising and [Person A] might have some struggles, we recommend moving the purifier to [Person A]'s room "
                            \item \textit{FR-NOTIF-4.3}: Location-specific example: "The ultrafine dust level in [Location] is high this afternoon, and I've detected some sneezing. I've activated 'Pollution Defense' mode to create a safe zone for you at home. Please remember to keep the windows closed! "
                            \item \textit{FR-NOTIF-4.4}: Predictive notifications must be sent 1-2 hours before predicted air quality deterioration
                            \item \textit{FR-NOTIF-4.5}: Include specific recommendations: window closure, room changes, device repositioning
                            \item \textit{FR-NOTIF-4.6}: Track prediction accuracy and adjust sensitivity thresholds based on user feedback
                        \end{enumerate}
                    \item Maintenance Notifications
                        \begin{enumerate}
                            \item \textit{FR-NOTIF-5.0}: Monitor filter life and send proactive replacement notifications
                            \item \textit{FR-NOTIF-5.1}: Filter warning at 90\% capacity: "Your filter is running low! Replace it soon to avoid air quality issues "
                            \item \textit{FR-NOTIF-5.2}: Critical filter notification at 100\% capacity: "Time for a fresh filter! I've reduced fan speed to protect the motor until you can replace it"
                            \item \textit{FR-NOTIF-5.3}: Include direct link to purchase replacement filters or schedule maintenance
                            \item \textit{FR-NOTIF-5.4}: Send reminder notifications every 3 days after initial filter warning
                            \item \textit{FR-NOTIF-5.5}: Other maintenance notifications: cleaning reminders, sensor calibration, software updates
                        \end{enumerate}
                    \item Notification Timing and Frequency
                        \begin{enumerate}
                            \item \textit{FR-NOTIF-6.0}: Respect user's local time zone and avoid notifications between 10 PM - 7 AM unless critical
                            \item \textit{FR-NOTIF-6.1}: Maximum 3 notifications per day per device to avoid notification fatigue
                            \item \textit{FR-NOTIF-6.2}: Priority system: Critical (filter/safety) >  Predictive > Responsive > General updates
                            \item \textit{FR-NOTIF-6.3}: User must be able to snooze non-critical notifications for 1hr, 4hr, or until tomorrow
                            \item \textit{FR-NOTIF-6.4}: Smart batching: combine multiple low-priority notifications into daily summary
                        \end{enumerate}
                    \item Notification Customization
                        \begin{enumerate}
                            \item \textit{FR-NOTIF-7.0}: Settings menu must include notification preferences with toggles for each notification type
                            \item \textit{FR-NOTIF-7.1}: Allow users to set "empathy level": Minimal, Standard, Caring, Very Caring
                            \item \textit{FR-NOTIF-7.2}: Empathy level affects message tone, frequency, and personalization depth
                            \item \textit{FR-NOTIF-7.3}: Quick notification response options: "Thanks!", "Not helpful", "Remind me later"
                            \item \textit{FR-NOTIF-7.4}: Learn from user responses to improve future notification relevance and timing
                        \end{enumerate}
                    \item Emergency and Safety Notifications
                        \begin{enumerate}
                            \item \textit{FR-NOTIF-8.0}: Override quiet hours for safety-critical notifications (device malfunction, air quality emergency)
                            \item \textit{FR-NOTIF-8.1}: Emergency example: "Air purifier has stopped working! Please check the device and ensure proper ventilation"
                            \item \textit{FR-NOTIF-8.2}: Severe air quality example: "Air quality in your area is hazardous. Please stay indoors and run the purifier on maximum"
                            \item \textit{FR-NOTIF-8.3}: Safety notifications must be persistent until acknowledged by user
                            \item \textit{FR-NOTIF-8.4}: Include emergency contact information or support links when appropriate
                        \end{enumerate}
                \end{enumerate}

\section{Development environment}
    \subsection{\textbf{Choice of Software Development Platform}}
        \begin{itemize}
            \item \textbf{Host OS}: Windows 11 Build 26200.6899/7019 
            \item \textbf{Language}: Python 3.11.9/3.12.5 for rapid prototyping, strong audio and ML ecosystem
            \item \textbf{Target Edge}: Ideal LG Purifier Hardware (for convenience).
        \end{itemize}
    \subsection{\textbf{Software in Use}}
        \begin{itemize}
            \item \textbf{ML}: TensorFlow, Scipy, librosa, scikit-learn.
            \item \textbf{Backend}: Firebase, Heroku
            \item \textbf{Database}: Firestore Database, PostgreSQL
            \item \textbf{DevOps}: VS Code, Git/GitHub, Git Projects, Docker
        \end{itemize}
        
    \begin{table}[!h] 
        \caption{Task Distribution}
            \begin{center}
                \small
                \setlength{\tabcolsep}{6pt}
                \renewcommand{\arraystretch}{1.3}
                    \begin{tabular}{|p{2cm}|p{1.5cm}|p{4cm}|}
                    \hline
                    \multicolumn{1}{|c|}{\textbf{Jobs}} & \multicolumn{1}{c|}{\textbf{Name}} & \multicolumn{1}{c|}{\textbf{Description}} \\ \hline
                    \centering Frontend & \centering Kim Soobin, Kim Yeonwoo & In charge of the planning, development and rollout of the PWA onto mobile devices. \\ \hline
                    \centering Backend & \centering Kim Yeonwoo & In charge of database and API creation, management, and maintainment \\ \hline
                    \centering ML/AI & \centering Jimmy MacDonald, Kim Minjin, et Al & In charge of collecting data, planning and creating an AI model to achieve the task at hand \\ \hline
                    \centering Project Manager & \centering Kim Minjin & Keeps track of documentation, assigns tasks, and make sure deadlines are met on time \\ \hline
                    \end{tabular}%
                \label{tab:task_distribution} % Added a specific label for reference
            \end{center}
        \end{table}

\section{Specification}
    \subsection{\textbf{User Authentication}}
        \begin{enumerate}
            \item Technology Stack
                \begin{itemize}
                    \item Frontend: React with Google OAuth 2.0 integration
                    \item Backend: Firebase Authentication via Admin SDK
                    \item Token Management: Google ID Tokens with sessionStorage persistence
                    \item Session Handling: Token-based authentication with automatic refresh
                \end{itemize}
            \item Authentication Flow
                \begin{enumerate}
                    \item User clicks "Login with Google"
                    \item User grants permissions
                    \item Frontend receives Google ID token
                    \item Token stored in sessionStorage as JSON: \{ "idToken": "...", "refreshToken": "...", "expiresIn": 3600 \}
                    \item Token included in Authorization header for all API requests
                    \item Backend validates token via Firebase Admin SDK
                    \item User authenticated and redirected to Home Page
                \end{enumerate}
            \item Token Verification Process
                \begin{enumerate}
                    \item Frontend retrieves token from localStorage: purecare\_auth
                    \item All API requests include header: Authorization: Bearer \{idToken\}
                    \item Backend validates using Firebase Admin SDK
                \end{enumerate}
            \item Security Measures
                \begin{itemize}
                    \item HTTPS-only communication in production
                    \item Token expiration: 1 hour (auto-refresh via Firebase SDK)
                    \item Secure sessionStorage (inaccessible to external scripts)
                    \item CORS configuration limits allowed origins to frontend URL only
                \end{itemize}
        \end{enumerate}
    \subsection{\textbf{Real-Time Device Control System}}
        \begin{enumerate}
            \item Communication Architecture
                \begin{itemize}
                    \item Protocol: WebSocket (Socket.io v4)
                    \item Transport: Bidirectional event-based communication
                    \item Fallback: HTTP long-polling if WebSocket unavailable
                    \item Connection: Persistent connection maintained per client
                \end{itemize}
            \item WebSocket Communication Flow
                \begin{enumerate}
                    \item Frontend
                    \item Backend
                    \item Device
                \end{enumerate}
            \item State Synchronization
                \begin{itemize}
                    \item Optimistic UI Updates: Frontend immediately reflects user changes 
                    \item Confirmation: Backend acknowledges command receipt within 200ms
                    \item Device Execution: Hardware simulator applies changes and confirms
                    \item Broadcast: Backend emits state update to all clients monitoring device
                    \item Timeout Handling: 3-second timeout with 2 automatic retries
                    \item Error Recovery: Rollback UI state if command fails
                \end{itemize}
            \item Controls
                \begin{itemize}
                    \item Power 
                        \begin{itemize}
                            \item type: "power"
                            \item value: true|false
                            \item deviceId
                            \item timestamp
                        \end{itemize}
                    \item Fan Speed 
                        \begin{itemize}
                            \item type: "fan\_speed"
                            \item value: 0-10
                            \item deviceId
                        \end{itemize}
                    \item Mode Selection
                        \begin{itemize}
                            \item type: "auto\_mode"
                            \item value: true|false
                            \item deviceId
                        \end{itemize}
                    \item Sensitivity
                        \begin{itemize}
                            \item type: "sensitivity"
                            \item value: "low" | "medium" | "high"
                        \end{itemize}
                \end{itemize}
        \end{enumerate}       
    \subsection{\textbf{Sensor Data Collection \& Time-Series Database}}
        \begin{enumerate}
            \item Database Architecture
                \begin{itemize}
                    \item Primary: PostgreSQL with TimescaleDB extension (Heroku Postgres Mini)
                    \item Schema: Hypertable optimized for time-series data
                    \item Aggregation: Continuous aggregates for hourly/daily summaries
                    \item Retention: 7 days raw data, 30 days hourly, 1 year daily
                \end{itemize}
        \end{enumerate}   
    \subsection{\textbf{Hardware Simulator (Python-based IoT Mock)}}
        \begin{enumerate}
            \item Purpose
                \begin{itemize}
                    \item In order to simulate our air purifier without having to purchase hardware, we made a python simulator to act as a mock device.
                \end{itemize}
            \item Simulation Logic
                \begin{itemize}
                    \item To initialize our variables, we take data from our cached database of the nearest station and set those as our active measurements.
                    \item Connect to our websocket so that our information shows up in the database and furthermore in our front end.
                    \item Send our current readings every \texttt{UPDATE\_INTERVAL} (default 15 seconds) through REST API.
                    \item If auto mode is on and if PM\textsubscript{2.5} is greater than 35, set the fan speed to the nearest integer corresponding to the total PM\textsubscript{2.5} / 10, with a maximum speed of 10
                    \item Reduction rate of particulate matter is determined on fan speed, calculated by 0.05 * fan speed. This is updated on each update interval.
                    \item Randomly events will appear based on the update interval.
                \end{itemize}
            \item Realistic Scenarios
                \begin{itemize}
                    \item To simulate a window being open, for each of these outside variables that aren't 0 (PM\textsubscript{2.5}, PM\textsubscript{10}, NO\textsubscript{2}), multiply the difference between the outdoor readings and the indoor readings by 0.15 and add it to the current sensor data
                    \item To simulate cooking, to each of these variables (PM\textsubscript{2.5}, PM\textsubscript{10}, NO\textsubscript{2}, TVOC, CO) we add a random uniform variable within norms of their respective scales.
                \end{itemize}
            \item Integration with Device Control
                \begin{itemize}
                    \item While the settings of the device can be controlled through the python script, updating its configuration in the front end will also update the simulator's values.
                \end{itemize}
        \end{enumerate}
    \subsection{\textbf{Automation Routine System}}
        \subsubsection{Routine Data Structure (Firebase Firestore)}
        \subsubsection{Execution Logic}
            \begin{enumerate}
                \item Based on a cloud function scheduled for certain times, run the user created function
            \end{enumerate}
        \subsubsection{Event Types}
    \subsection{\textbf{Notification System with Empathetic Intelligence}}
        \subsubsection{Notification Types}
        \subsubsection{Personalization Engine}
        \subsubsection{Delivery System}
        \subsubsection{Rate Limiting}
    \subsection{\textbf{Audible Health Symptom Detection}}
        \begin{enumerate}
            \item System Architecture: 
                The PureCare Air Purifier implements a two-stage neural network architecture for intelligent cough detection and classification, replacing traditional threshold-based approaches with sophisticated machine learning inference.
                \begin{enumerate}
                    \item Cough Detection Model
                        \begin{itemize} 
                            \item Architecture: Lightweight 1D Convolutional Neural Network (CNN) 
                            \item Task: Binary classification (cough vs. non-cough) 
                            \item Input: 39-dimensional MFCC features over 1.5-second audio windows 
                            \item Output: Sigmoid activation producing probability score [0,1] \item Decision threshold: 0.7 confidence for positive detection 
                            \item Design priority: Real-time inference speed and low computational overhead 
                        \end{itemize}
                    \item Cough Classification Model
                        \begin{itemize} 
                            \item Architecture: Hybrid CNN-LSTM network 
                            \item Task: Multi-class classification of cough types 
                            \item Purpose: Identifies specific cough characteristics for health insights 
                            \item Output: Softmax activation over N cough categories \item Temporal modeling: LSTM layer captures sequential dependencies in cough patterns 
                            \item Activation: Only processes audio segments flagged by Stage 1 detection 
                        \end{itemize}
                \end{enumerate}
            \item Feature Extraction
                \begin{itemize}
                    \item MFCC (Mel-Frequency Cepstral Coefficients):
                        The system employs MFCC feature extraction, a standard technique in audio signal processing that mimics human auditory perception
                         \begin{itemize} 
                            \item Sample Rate: 8kHz (optimized for low-frequency respiratory sounds) 
                            \item Window Size: 1.5 seconds (captures complete cough events) 
                            \item Base MFCCs: 13 coefficients 
                            \item Delta Features: 13 first-order derivatives (velocity of spectral change) 
                            \item Delta-Delta Features: 13 second-order derivatives (acceleration of spectral change) 
                            \item Total Feature Dimensions: 39 per time frame 
                            \item FFT Window: 512 samples with 256-sample hop length 
                            \item Mel Filterbank: 40 triangular filters spanning 50Hz to 4000Hz    
                        \end{itemize}
                    \item Feature Normalization
                    \begin{itemize}
                        \item All extracted features undergo zero-mean unit-variance normalization to improve neural network training stability and convergence: $$\text{normalized} = \frac{\text{features} - \mu}{\sigma}$$ where $\mu$ and $\sigma$ are computed per-feature across the time dimension
                    \end{itemize}
                \end{itemize}
            \item Live Detection Pipeline
                \begin{itemize}
                    \item Audio Acquisition and Buffering
                        \begin{enumerate} 
                            \item Continuous audio stream captured at 8kHz, mono channel 
                            \item Circular buffer maintains 10 seconds of recent audio (80,000 samples) 
                            \item Thread-safe deque structure with mutex-protected access 
                            \item Block size: 2048 samples for efficient callback processing
                        \end{enumerate}
                    \item Sliding Window Detection
                        \begin{enumerate} 
                            \item Detection Window: 1.5 seconds (12,000 samples) 
                            \item Hop Length: 0.5 seconds (4,000 samples) — 66\% overlap between windows 
                            \item Processing Rate: Approximately 2 windows per second 
                            \item Overlap strategy: Ensures no cough events are missed between windows 
                        \end{enumerate}
                    \item Energy Pre-filtering (Optional Optimization): 
                        Before neural network inference, a fast RMS energy calculation filters silent segments: $$\text{RMS} = \sqrt{\frac{1}{N}\sum_{i=1}^{N} x_i^2}$$
                        \begin{itemize} 
                            \item Energy Threshold: 0.01 (configurable) 
                            \item Purpose: Reduces unnecessary NN inference on silence, improving efficiency 
                            \item Computational cost: O(N) vs. O(N log N) for MFCC extraction
                        \end{itemize}
                    \item Neural Network Inference
                        \begin{enumerate} 
                            \item Extract MFCC features from 1.5s window: Shape (39, 47 frames)
                            \item Stage 1: Pass features through detection CNN 
                            \item If detection confidence >= 0.7: Flag as cough event 
                            \item Stage 2: Pass same features through classification CNN-LSTM 
                            \item Extract cough type/category and confidence score 
                            \item Log detection with timestamp, confidence, and classification
                        \end{enumerate}
                \end{itemize}
            \item Training and Data Augmentation
                \begin{itemize}
                    \item Dataset Preparation
                        \begin{itemize} 
                            \item Train/Validation/Test Split: 70\%/15\%/15\% 
                            \item Fixed random seed for reproducible splits 
                            \item Stratified sampling maintains class balance across splits 
                        \end{itemize}
                    \item Data Augmentation Techniques
                        To improve model generalization and robustness to real-world variations: 
                        \begin{enumerate} 
                            \item Time Shifting: Circular shift of up to ±20\% of signal length 
                            \item Speed Perturbation: Time-stretching with rate factor in [0.9, 1.1] 
                            \item Additive Noise: White noise or environmental noise at SNR between -5dB and 15dB 
                            \item Application: Random combination of augmentations during training only 
                        \end{enumerate}
                    \item Training Configuration
                        \textbf{Detection Model:} 
                            \begin{itemize} 
                                \item Optimizer: Adam with learning rate 0.001 
                                \item Loss Function: Binary cross-entropy 
                                \item Batch Size: 32 
                                \item Epochs: 50 (with early stopping, patience=10) 
                                \item Learning Rate Schedule: ReduceLROnPlateau (factor=0.5, patience=5) 
                                \item Metrics: Accuracy, Precision, Recall, AUC 
                            \end{itemize} 
                        \textbf{Classification Model:} 
                            \begin{itemize} 
                                \item Optimizer: Adam with learning rate 0.0001 
                                \item Loss Function: Categorical cross-entropy 
                                \item Batch Size: 32 
                                \item Epochs: 100 (with early stopping, patience=15) 
                                \item Learning Rate Schedule: ReduceLROnPlateau (factor=0.5, patience=7) 
                                \item Metrics: Categorical accuracy, Top-2 accuracy 
                            \end{itemize}
                \end{itemize}
            \item Model Architecture Details
                \begin{itemize}
                    \item Detection Model (1D CNN)
                        \begin{enumerate} 
                            \item Input: (39 features, 47 time frames) 
                            \item Permutation: Transpose to (47 time frames, 39 features) for 1D convolution 
                            \item Conv1D Block 1: 32 filters, kernel=3, ReLU → BatchNorm → MaxPool(2) → Dropout(0.3)
                            \item Conv1D Block 2: 64 filters, kernel=3, ReLU → BatchNorm → MaxPool(2) → Dropout(0.3)
                            \item Conv1D Block 3: 128 filters, kernel=3, ReLU → BatchNorm → MaxPool(2) → Dropout(0.3)
                            \item Global Average Pooling: Reduces temporal dimension 
                            \item Dense: 64 units, ReLU → Dropout(0.3) 
                            \item Output: 1 unit, Sigmoid activation 
                            \item Total Parameters: ~50K (lightweight for edge deployment) 
                        \end{enumerate}
                    \item Classification Model (CNN-LSTM)
                        \begin{enumerate} 
                            \item Input: (39 features, 47 time frames) 
                            \item Permutation: Transpose to (47 time frames, 39 features) 
                            \item Conv1D Block 1: 64 filters, kernel=3, ReLU → BatchNorm → MaxPool(2) → Dropout(0.4)
                            \item Conv1D Block 2: 128 filters, kernel=3, ReLU → BatchNorm → MaxPool(2) → Dropout(0.4)
                            \item Conv1D Block 3: 256 filters, kernel=3, ReLU → BatchNorm → MaxPool(2) → Dropout(0.4)
                            \item LSTM Layer: 128 units (captures temporal dependencies in cough dynamics) 
                            \item Dropout(0.4) 
                            \item Dense: 128 units, ReLU → Dropout(0.4) 
                            \item Output: N units (number of cough classes), Softmax activation 
                            \item Total Parameters: ~400K 
                        \end{enumerate}
                \end{itemize}
        \end{enumerate}
    \subsection{\textbf{Cost Analysis \& Infrastructure}}
            \begin{table}[htbp]
            \caption{Heroku Deployment Costs}
            \begin{center}
            \small
            \setlength{\tabcolsep}{4pt} 
            \renewcommand{\arraystretch}{1.3} 
            \begin{tabular}{|p{1.8cm}|p{1.5cm}|p{1.8cm}|p{2.5cm}|}
            \hline
            \multicolumn{1}{|c|}{\textbf{Service}} & \multicolumn{1}{c|}{\textbf{Plan}} & \multicolumn{1}{c|}{\textbf{Monthly Cost}} & \multicolumn{1}{c|}{\textbf{Features}} \\ \hline
            Web Dyno & Hobby & \$7 & Always-on, 512MB RAM \\ \hline
            Postgres & Essential & \$5 & 10GB storage, TimescaleDB \\ \hline
            \textbf{Total} & & \textbf{\$12/month} & 26 months with \$312 credit \\ \hline
            \end{tabular}%
            \label{tab:heroku_costs}
            \end{center}
            \end{table}
        \subsubsection{Data Volume Estimates}
        \begin{itemize}
            \item Per Device
            \begin{itemize}
                \item 8 sensors × 288 readings/day (every 5 min) = 2,304 records/day
                \item 7 days raw data = ~16,000 records = \~1MB
            \end{itemize}
        \end{itemize}
        \subsubsection{Performance Metrics}
        \begin{itemize}
            \item WebSocket latency <100ms
            \item Sensor data processing: <50ms
            \item Database query time: <200ms (with TimescaleDB indexes)
            \item Frontend load time: <2s
        \end{itemize}
            
\section{Architecture Design \& Implementation}
    \subsection{Overall Architecture}
    \begin{figure}
        \centering
        \includegraphics[width=1\linewidth]{NetworkArchitecture.png}
        \caption{Network Architecture}
        \label{fig:placeholder}
    \end{figure}
    \subsection{Directory Organization}
        \begin{enumerate}
                \begin{table}[htbp]
                    \caption{Front end directory}
                    \begin{center}
                        \footnotesize
                        \setlength{\tabcolsep}{4pt} 
                        \renewcommand{\arraystretch}{1.3} 
                            \begin{tabular}{|p{2.1cm}|p{1.4cm}|p{1cm}|p{2cm}|}
                                \hline
                                \multicolumn{1}{|c|}{\textbf{Directory}} & 
                                \multicolumn{1}{c|}{\textbf{Key Files}} & 
                                \multicolumn{1}{c|}{\textbf{Module Components}} & 
                                \multicolumn{1}{c|}{\textbf{Description}} \\ \hline
                                \path{/client/src/app/[locale]/(auth)/} &
                                \path{login/page.tsx} \path{signup/page.tsx} &
                                User Authentication UI & 
                                Google OAuth 2.0 login/signup screens (FR-002) \\ \hline
                                \path{/client/src/app/[locale]/(core)/} & 
                                \path{home/page.tsx} & 
                                Home Page & 
                                Main dashboard with device list, AQI display, user greeting (FR-004) \\ \hline
                                \path{/client/src/app/[locale]/automation/} &
                                \path{page.tsx} & 
                                Automation/Routine System UI & 
                                Routine management screen with routine cards (FR-006) \\ \hline
                                \path{/client/src/app/[locale]/room/[id]/} &
                                \path{page.tsx} &
                                Device Control Page &
                                Real-time device control interface with power, fan speed, timer, child lock controls (FR-008) \\ \hline
                                \path{/client/src/app/[locale]/settings/} &
                                \path{page.tsx} \path{account/page.tsx}
                                \path{devices/page.tsx}
                                \path{location/page.tsx}
                                \path{privacy/page.tsx} &
                                Settings Module &
                                Account settings, device management, location selection, privacy controls (FR-009 to FR-013) & \\ \hline
                                \path{/client/src/app/[locale]/devices/add/} &
                                \path{page.tsx}
                                \path{qr/page.tsx}
                                \path{qr/confirm/page.tsx}
                                \path{serial/page.tsx}
                                \path{serial/success/page.tsx} &
                                Device Registration &
                                QR code scanning and serial number entry for device pairing (FR-003) \\ \hline
                                \path{/client/src/app/[locale]/report/} &                  
                                \path{page.tsx} &
                                Data Visualization &
                                Sensor data reports and analytics dashboard \\ \hline
                                \path{/client/src/app/[locale]/weather/} &
                                \path{page.tsx} &
                                Weather Integration &
                                External AQI and weather data display interface \\ \hline
                                \path{/client/src/app/api/} &
                                \path{forecast/route.ts}
                                \path{geocode/route.ts}
                                \path{weather/route.ts} &
                                API Routes &
                                Next.js API endpoints for weather and geocoding services \\ \hline
                            \end{tabular}%
                        \end{center}
                    \end{table}
                    \begin{table}[htbp]
                    \caption{Front end directory contd.}
                    \begin{center}
                        \footnotesize
                        \setlength{\tabcolsep}{4pt} 
                        \renewcommand{\arraystretch}{1.3} 
                            \begin{tabular}{|p{2.1cm}|p{1.4cm}|p{1cm}|p{2cm}|}
                                \hline
                                \multicolumn{1}{|c|}{\textbf{Directory}} & 
                                \multicolumn{1}{c|}{\textbf{Key Files}} & 
                                \multicolumn{1}{c|}{\textbf{Module Components}} & 
                                \multicolumn{1}{c|}{\textbf{Description}} \\ \hline
                                
                                \path{/client/src/components/features/} &
                                \seqsplit{aqi-trend-chart.tsx} 
                                \seqsplit{device-carousel.tsx}
                                \seqsplit{kakao-map.tsx}
                                \seqsplit{welcome-modal.tsx} &
                                Feature Components &
                                Reusable UI components for charts, device displays, maps \\ \hline
                                
                                \path{/client/src/components/layout/} &
                                \seqsplit{bottom-nav.tsx}
                                \seqsplit{client-layout-wrapper.tsx}
                                \seqsplit{splash.tsx} &
                                Layout Components &
                                Navigation bar (FR-005), splash screen, layout wrappers \\ \hline
                                
                                \path{/client/src/components/ui/} &
                                \seqsplit{demo-mode-banner.tsx}
                                \seqsplit{segmented-control.tsx} &
                                UI Components &
                                Shared UI elements like segmented controls, banners \\ \hline
                                
                                \path{/client/src/lib/} &
                                \path{api.ts}
                                \path{auth.tsx}
                                \path{firebase.ts}
                                \path{fcm.ts}
                                \path{weather.tsx} &
                                Client Libraries &
                                API client, authentication logic, Firebase config, FCM push notifications (FR-014), weather API integration \\ \hline
                                
                                \path{/client/src/i18n/} &
                                \path{request.ts}
                                \path{routing.ts} &
                                Internationalization &
                                Multi-language support (i18n) routing and request handling \\ \hline
                                
                        \end{tabular}%
            \end{center}
        \end{table}
            \begin{table}[htbp]
                \caption{Backend directory}
                \begin{center}
                    \footnotesize
                    \setlength{\tabcolsep}{4pt} 
                    \renewcommand{\arraystretch}{1.3} 
                        \begin{tabular}{|p{2.1cm}|p{1.4cm}|p{1cm}|p{2cm}|}
                            \hline
                            \multicolumn{1}{|c|}{\textbf{Directory}} & 
                            \multicolumn{1}{c|}{\textbf{Key Files}} & 
                            \multicolumn{1}{c|}{\textbf{Module Components}} & 
                            \multicolumn{1}{c|}{\textbf{Description}} \\ \hline
                            \path{/server/src/api/} &
                            \path{index.js} &
                            Express Server Entry &
                            Main server initialization with Socket.io WebSocket integration \\ \hline
                            \path{/server/src/api/config/} & 
                            \path{firebase.js} & 
                            Firebase Configuration & 
                            Firebase Admin SDK initialization for authentication \\ \hline
                            \path{/server/src/api/database/} &
                            \path{init.js} & 
                            Database Initialization & 
                            PostgreSQL / TimescaleDB connection setup \\ \hline
                            \path{/server/src/api/middleware/} &
                            \path{auth.js}
                            \seqsplit{rateLimiter.js} &
                            Authentication \& Security &
                            JWT token verification middleware, API rate limiting \\ \hline
                            \path{/server/src/api/routes/} &
                            \seqsplit{auth.js}
                            \seqsplit{controlRoutes.js}
                            \seqsplit{deviceRoutes.js}
                            \seqsplit{roomRoutes.js}
                            \seqsplit{sensorRoutes.js}
                            \seqsplit{userRoutes.js}
                            \seqsplit{notificationRoutes.js} &
                            REST API Endpoints &
                            Authentication, device control commands, device registration, room management, sensor data queries, user profile, push notifications \\ \hline
                            \path{/server/src/api/services/} &
                            \seqsplit{deviceService.js}
                            \seqsplit{sensorDataService.js}
                            \seqsplit{notificationService.js}
                            \seqsplit{timezoneService.js} &
                            Business Logic Services &
                            Device state management, time-series data aggregation, empathetic notification generation (FR-014), timezone handling \\ \hline
                            \path{/server/src/api/scripts/} &
                            \seqsplit{aqiScripts.js}
                            \seqsplit{midnightRoutine.js}
                            \seqsplit{processNotifications.js} &
                            Background Jobs \& Scripts &
                            AQI data fetching, daily reset routines, notification batch processing \\ \hline
                        \end{tabular}%
                \end{center}
            \end{table}
            
            \begin{table}[htbp]
                \caption{Hardware simulator directory}
                \begin{center}
                    \footnotesize
                    \setlength{\tabcolsep}{4pt} 
                    \renewcommand{\arraystretch}{1.3} 
                        \begin{tabular}{|p{2.1cm}|p{1.4cm}|p{1cm}|p{2cm}|}
                            \hline
                            \multicolumn{1}{|c|}{\textbf{Directory}} & 
                            \multicolumn{1}{c|}{\textbf{Key Files}} & 
                            \multicolumn{1}{c|}{\textbf{Module Components}} & 
                            \multicolumn{1}{c|}{\textbf{Description}} \\ \hline
                            \path{/hardware/simulator/} &
                            \path{__main__.py}
                            \path{run_simulator.py} &
                            Simulator Entry Point &
                            Main Python script to launch air purifier simulator \\ \hline
                            \path{/hardware/simulator/core/} & 
                            \path{device.py}
                            \path{audio_detector_wrapper.py} & 
                            Device Simulation Logic & 
                            Air purifier behavior simulation: PM reduction rate (0.05 $\times$ fan\_speed), auto mode logic (PM\textsubscript{2.5} $>$ 35), sensor value updates every 15s \\ \hline
                            \path{/hardware/simulator/models/} &
                            \path{commands.py}
                            \path{responses.py}
                            \path{sensor_data.py} & 
                            Data Models & 
                            Command structures (power, fan\_speed, auto\_mode, sensitivity), response formats, sensor reading schemas \\ \hline
                            \path{/hardware/simulator/communication/} &
                            \path{websocket_client.py}
                            \path{http_client.py}
                            \path{event_sender.py} &
                            Real-Time Device Control &
                            WebSocket client for bidirectional Socket.io communication, REST API client for sensor data upload, event emission for cough/sneeze detection \\ \hline
                            \path{/hardware/simulator/config/} &
                            \path{settings.py}
                            \path{constants.py} &
                            Configuration &
                            Update intervals (default 15s), AQI station data, device serial numbers, server URLs \\ \hline
                            \path{/hardware/shared/events/} &
                            \path{bus.py} &
                            Event System &
                            Event bus for inter-process communication between simulator and audio detector \\ \hline
                            \path{/hardware/} &
                            \path{cli_controller.py}
                            \path{send_command.py} &
                            CLI Control &
                            Command-line interface for manual device control \\ \hline
                        \end{tabular}%
                \end{center}
            \end{table}
            
            \begin{table}[htbp]
                \caption{ML/AI audio detection directory}
                \begin{center}
                    \footnotesize
                    \setlength{\tabcolsep}{4pt} 
                    \renewcommand{\arraystretch}{1.3} 
                        \begin{tabular}{|p{2.1cm}|p{1.4cm}|p{1cm}|p{2cm}|}
                            \hline
                            \multicolumn{1}{|c|}{\textbf{Directory}} & 
                            \multicolumn{1}{c|}{\textbf{Key Files}} & 
                            \multicolumn{1}{c|}{\textbf{Module Components}} & 
                            \multicolumn{1}{c|}{\textbf{Description}} \\ \hline
                            \path{/hardware/audio_analyzer/} &
                            \path{train.py}
                            \path{evaluate_model.py}
                            \path{live_detection.py} &
                            Model Training \& Inference &
                            Train detection/classification CNNs, evaluate model performance, real-time audio detection pipeline \\ \hline
                            \path{/hardware/audio_analyzer/} & 
                            \path{models.py} & 
                            Neural Network Architecture & 
                            Stage 1: 1D CNN for binary cough detection (39 MFCCs, 0.7 confidence threshold, $\sim$50K params). Stage 2: CNN-LSTM for cough classification (128 LSTM units, $\sim$400K params) \\ \hline
                            \path{/hardware/audio_analyzer/} &
                            \seqsplit{feature_extraction.py} & 
                            Audio Feature Engineering & 
                            MFCC extraction (8kHz, 13 base + 13 delta + 13 delta-delta = 39 features), 1.5s windows with 0.5s hop length, zero-mean unit-variance normalization \\ \hline
                            \path{/hardware/audio_analyzer/} &
                            \path{data_loader.py}
                            \path{prepare_dataset.py}
                            \path{download_audeering_dataset.py} &
                            Dataset Management &
                            70/15/15 train/val/test split, stratified sampling, data augmentation (time shifting, speed perturbation, additive noise) \\ \hline
                            \path{/hardware/audio_analyzer/hardware/AI/models/} &
                            \path{detection_model_best.h5}
                            \path{classification_model_best.h5}
                            \path{*.json} &
                            Trained Models \& Metadata &
                            Best detection model checkpoint, best classification model checkpoint, training history logs, fixed data split configuration \\ \hline
                            \path{/hardware/audio_analyzer/hardware/AI/cough_dataset/} &
                            \path{cough/}
                            \path{non_cough/}
                            \path{sneeze/}
                            \path{speech/} &
                            Training Dataset &
                            Labeled audio samples for model training (multi-class categories) \\ \hline
                            \path{/hardware/tools/} &
                            \path{audio_capture.py}
                            \path{playback.py}
                            \path{segmentation.py} &
                            Audio Utilities &
                            Audio recording tools, playback utilities, audio segmentation for dataset preparation \\ \hline
                        \end{tabular}%
                \end{center}
            \end{table}
        \end{enumerate}
    \subsection{Next.js Frontend}
        For the PureCare Air Purifier to have empathetic intelligence, we needed a way to deliver this sentiment to the user. To do this, we made a Next.js based Progressive Web App. The app allows us to communicate and receive inputs from users, making for a truly responsive design. Within the app, users are able to register, control, and receive real time data from the air purifier.
        The source code for the front end can be found in the \texttt{client} folder.
    \subsection{Express Backend}
        In order to fulfill key essential functions for the device, we needed to design some sort of server to support our devices and front end. To resolve this issue, we designed a fully custom express.js backend to power our services. The backend, which is hosted on Heroku, allows us to be a middleman between the physical hardware and the user software, while also giving crucial information for the hardware so that it can act proactively rather than reactively.
    \subsection{Hardware}
        As the main product for our service, the hardware serves as the very heart of our project. Our hardware serves to build off pre existing \textit{LG PuriCare Air Purifiers} and enhance them to provide a better experience for users. 

\section{Use Cases}
    \subsection{Use Case 1}
        \begin{enumerate}
            \item 1st instruction
            \item 2nd Instruction
        \end{enumerate}
    \subsection{Use case 2}
\section{Software Installation Guide}

\section{Discussion}
One of the biggest issues in this project was the language barrier which lead to many problems. Clear communication was a struggle and keeping track of what each other had done, despite the use of the github projects, was a real issue.

\begin{thebibliography}{00}
\bibitem{b1}  World Health Organization, "Billions of people still breathe unhealthy air: new WHO data," WHO News, 04-Apr-2022. [Online]. Available: https://www.who.int/news/item/04-04-2022-billions-of-people-still-breathe-unhealthy-air-new-who-data  (Accessed: 01-Oct-2025).
\bibitem{b2} IEEE S. Amiriparian et al., "CAST a database: Rapid targeted large-scale big data acquisition via small-world modelling of social media platforms," in 2017 Seventh International Conference on Affective Computing and Intelligent Interaction (ACII), 2017, pp. 340–345, doi: 10.1109/ACII.2017.8273622.

\end{thebibliography}
\vspace{12pt}

\end{document}