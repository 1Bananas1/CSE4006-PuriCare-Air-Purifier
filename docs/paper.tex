\documentclass[conference]{IEEEtran}
\IEEEoverridecommandlockouts
% The preceding line is only needed to identify funding in the first footnote. If that is unneeded, please comment it out.
\usepackage{cite}
\usepackage{amsmath,amssymb,amsfonts}
\usepackage{algorithmic}
\usepackage{graphicx}
\usepackage{comment}
\usepackage{textcomp}
\usepackage{fixltx2e}
\usepackage{xcolor}
\usepackage{enumitem}
\setlist[enumerate,1]{label=\Alph*., leftmargin=25pt, itemsep=8pt}
\setlist[enumerate,2]{label=\arabic*), leftmargin=20pt, itemsep=3pt}
\setlist[itemize]{label=\textbullet, leftmargin=20pt, itemsep=2pt}
\def\BibTeX{{\rm B\kern-.05em{\sc i\kern-.025em b}\kern-.08em
    T\kern-.1667em\lower.7ex\hbox{E}\kern-.125emX}}
\begin{document}

\title{PureCare Air Purifier\\

\thanks{}
}

\author{
\IEEEauthorblockN{Jimmy MacDonald}
\IEEEauthorblockA{\textit{dept. of Computer Science} \\
\textit{Hanyang University}\\
Seoul, South Korea \\
james.macdonald.1@slu.edu}
\and
\IEEEauthorblockN{Yimsoobeen}
\IEEEauthorblockA{\textit{dept. of Information System}\\
\textit{Hanyang University}\\
Seoul, South Korea \\
s00been@hanyang.ac.kr}
\and
\IEEEauthorblockN{Kim Minjin}
\IEEEauthorblockA{\textit{dept. of Information System} \\
\textit{Hanyang University}\\
Seoul, South Korea \\
idid02@hanyang.ac.kr}
\and
\IEEEauthorblockN{Yeonwoo Kim}
\IEEEauthorblockA{\textit{dept. of Information System} \\
\textit{Hanyang University}\\
Seoul, South Korea \\
bbongvong@hanyang.ac.kr}
}

\maketitle

\begin{abstract}
Air purifiers have become more and more common in people’s houses, but as houses gain more technology, it seems that air purifiers haven’t had any major improvements. We propose a smart air purifier which learns and adapts how it runs based on it’s users. Specifically, by monitoring and recognizing when users are sick, gaining insights about air quality outside as well as seasonal allergies, and smart scheduling on when to run the air purifier to minimize electricity usage.
\end{abstract}

\begin{IEEEkeywords}
Smart Air Purifier, AQI
\end{IEEEkeywords}

\begin{table}[htbp]
\caption{Role Assignments}
\begin{center}
\small
\setlength{\tabcolsep}{6pt} 
\renewcommand{\arraystretch}{1.3} 
\begin{tabular}{|p{2cm}|p{1.5cm}|p{4cm}|}
\hline
\multicolumn{1}{|c|}{\textbf{Roles}} & \multicolumn{1}{c|}{\textbf{Name}} & \multicolumn{1}{c|}{\textbf{Task description and etc.}} \\ \hline
\centering User & \centering Yim Soobin & Identifies user needs and improves product usability. Thinks of ideas, designs functions, and suggests practical ways to implement features that address real-world problems. \\ \hline
\centering Customer & \centering Jimmy MacDonald & Conducts competitive analysis and plans differentiation strategy. Examines competitor services, identifies market gaps, and designs UI elements that attract customers. \\ \hline
\centering Software developer & \centering Kim Yeonwoo & Converts requirements into functional software. Writes code, implements features, debugs programs, monitors performance, and ensures all components execute properly. \\ \hline
\centering Development Manager & \centering Kim Minjin & Manages overall development process and team coordination. Creates plans, assigns roles, monitors progress, manages schedules, and ensures smooth communication among team members. \\ \hline
\end{tabular}%
\label{tab1}
\end{center}
\end{table}

\newpage

\section{Introduction}
\subsection{Motivation}
As the world continues to industrialize, air quality has declined as a direct consequence. As harmful particulate matter such as PM\textsubscript{2.5}, PM\textsubscript{10}, and other volatile organic matter enter the atmosphere, residents find difficulty in breathing under these conditions. The World Health Organization (WHO), estimates that 99\% of the global population breathes air that exceeds guidelines, leading to millions of premature deaths annually from various diseases \cite{b1}. 

Despite living in a technology-driven age, air purifiers have seemed to lack any significant advances, remaining more reactive than proactive. Modern air purifiers activate only once particles have been detected in the air, leaving users exposed to a harmful environment until the device responds. While this reactive approach provides some benefit, the delay between exposure and activation can lead to numerous audible health symptoms, such as sneezing or coughing, potentially interrupting important moments such as presentations or meeting. Furthermore, current air purifiers require human configuration and continue to run continuously regardless if anyone is home resulting in a waste of electricity.

We propose PureCare, a truly intelligent air purification system that acts as a proactive health assistant rather than a passive appliance. PureCare distinguishes itself through several key innovations: Audible Health Symptom Recognition technology that detects sneezing, coughing, and snoring to automatically adjust purification settings; geofencing capabilities that optimize operation based on occupancy; API integration with local air quality data to prepare for poor outdoor conditions; personalized care recommendations that learn individual sensitivities over time; and sleep mode optimization for nighttime respiratory issues. Unlike traditional purifiers that simply filter air, PureCare anticipates needs, provides gentle human-like notifications, and creates safer environments before problems arise.
\newpage
\subsection{Similar Products}
    \subsubsection{LG PuriCare Objet Collection AI+ 360˚ Air Purifier}
        \begin{itemize}
            \item Detects harmful gas and fumes from three major sick-building substances-formaldehyde ammonia, and volatile oraganinc compounds (VOCs)-and automatically purifies the air according to gas type and pollution level.
            \item Detects ammonia, the main source of pet waste odors.
            \item Detects contaminants such as cooking fumes.
            \item “AI Customized Operation” learns and analyzes indoor air quality every hour. Using accumulated data, it identifies times when the air is clean, stops the purifier’s fan, and dims the display to save energy.
            \item By autonomously analyzing air quality and adjusting operating intensity, it can reduce power consumption by up to 50 percent or more compared with the conventional AI mode, helping lower electricity costs.
        \end{itemize}
    \subsubsection{SAMSUNG BESPOKE Cube Air Infinite Line}
        \begin{itemize}
            \item Provides an AI-powered integrated personalized cleaning solution.
            \item “Customized Clean AI+” compares and learns indoor and outdoor air quality, and pre-cleans when deterioration is predicted.
            \item “AI Saving Mode” automatically adjusts airflow or stops the fan when indoor air improves, cutting energy use by up to 45% on the 100㎡ model.
            \item “Customized Clean AI+” is certified with “AI+ Certification” by the Korea Standards Association.
        \end{itemize}
    \subsubsection{SK Wellness Robot}
        \begin{itemize}
            \item The Wellness Robot by SK Namoothix’s brand, NAMUHX, removes harmful substances that enter during ventilation to maintain air quality. It also offers performance equivalent to six air purifiers. All of its air solution functions are controlled entirely by voice without touch, and through the AI control system, device monitoring and remote support for maintenance are provided, greatly enhancing user convenience.

        \end{itemize}






\section{Requirement Analysis}

\subsection{\textbf{App Installation and First Launch}}

\subsubsection{FR-001: Application and First Launch}

\begin{enumerate}
\item Platform Availability
    \begin{enumerate}
    \item iOS Platform: Available on App Store for iOS 12.0+ (iPhone 6S and newer)
    \item Android Platform: Available on Google Play Store for Android 10+ (Armv7 and ARM64 architectures)
    \end{enumerate}

\item First Launch Experience
    \begin{enumerate}
    \item Splash Screen: Displays application logo and version number for 1-2 seconds
    \item Permission Requests: Camera (for QR scanning), Notifications (for device alerts), and optional Location (for weather recommendations)
    \end{enumerate}
\end{enumerate}

\subsection{\textbf{Login/Sign Up}}

\subsubsection{FR-002: User Authentication}

\begin{enumerate}
\item Welcome Screen\\
Displays after splash screen with application logo, tagline, and two buttons: "Login" and "Sign Up"

\item Login Screen\\
Contains email and password fields, "Remember me" checkbox, "Forgot Password?" link, and "Don't have an Account? Sign Up" link

\item Sign-up Flow
    \begin{enumerate}
    \item Page 1: Basic account information (Full Name, Email, Password, Confirm Password, Terms Agreement)
    \item Page 2: Optional profile information (Phone, Profile Picture, DOB, Country)
    \end{enumerate}
\end{enumerate}

\subsection{\textbf{Product Setup via QR Code}}

\subsubsection{FR-003: Product Registration}

\textit{Note: Only accessible after successful login/sign up}

\begin{enumerate}
\item Access Points
    \begin{enumerate}
    \item Automatically shown to new users after login (if no devices registered)
    \item "Add Device" button on Home Page
    \item "+" button in navigation bar
    \end{enumerate}

\item Initial Screen for New Users\\
If user has zero registered devices, shows empty state with title "Connect your first device", description "Scan the QR code on your device to get started", "Scan QR Code" button, and "Enter serial number manually" link

\item QR Code Scanning Process
    \begin{enumerate}
    \item Camera Activation: Checks auth token validity. If invalid, redirects to login with message "Please login to register devices"
    \item QR Detection and Validation: Scans QR code, validates against database, checks if device already registered. Error shown: "This device is registered to another account"
    \item Device Linking: Links device to user account, stores User ID, Device Serial Number, Device Model, and Registration Timestamp
    \end{enumerate}

\item Product Information Confirmation\\
Displays product image, model name, serial number, and user's name with prompt "Register this device to \{Name\}'s account?" with confirm/cancel buttons

\item Registration Complete\\
Shows success message "Device registered successfully", sends email notification, offers "Go home" or "Add another device" options, then navigates to Home Page
\end{enumerate}

\subsection{\textbf{Home Page}}

\subsubsection{FR-004: Home Page}

\textit{Note: Only accessible after successful login/sign up}

\begin{enumerate}
\item Home Page Access\\
Displayed after successful login, device registration (first-time), or tapping "home" tab. Checks authentication on load—expired tokens redirect to Login

\item Home Page Layout
    \begin{enumerate}
    \item Header: User greeting "Welcome back \{user\}", current AQI conditions, notification bell, settings gear
    \item Device List: Shows all registered devices or "No devices yet..." if empty. Each card displays Device Name, Model, Status, Last Active time, and Quick Action button. Includes "add device" floating action button
    \end{enumerate}

\item Device Status Synchronization\\
Loads device list from database, queries IoT server status, updates UI with real-time data, auto-refreshes every 30 seconds
\end{enumerate}

\subsection{\textbf{Navigation Bar}}

\subsubsection{FR-005: Navigation Bar}

\begin{enumerate}
\item Bottom Navigation Structure\\
Persistent across all main screens with four tabs: Home (house icon), Dashboard (grid icon), Control (remote icon), Account (profile icon). When logged out, tapping any tab redirects to login
\end{enumerate}

\subsection{\textbf{Dashboard}}

\subsubsection{FR-006: Dashboard}

\textit{Note: Only accessible after successful login/sign up}

\begin{enumerate}
\item Dashboard Data Scope\\
Displays analytics for all devices registered to current user (Usage History, Active Status Graph), filtered by user ID from auth token

\item Dashboard Sections
    \begin{enumerate}
    \item Summary Statistics: Total Devices count, Active Devices count, Cumulative Usage Time
    \item Device Overview: Filter health status and quick controls for each device
    \end{enumerate}
\end{enumerate}

\subsection{\textbf{Control Page}}

\subsubsection{FR-007: Device Control Page}

\begin{enumerate}
\item Access Control\\
Manages user permissions for device control operations

\item Device Control Interface\\
Provides interface for controlling individual air purifier settings and operations
\end{enumerate}

\begin{comment}
The IEEEtran class file is used to format your paper and style the text. All margins, 
column widths, line spaces, and text fonts are prescribed; please do not 
alter them. You may note peculiarities. For example, the head margin
measures proportionately more than is customary. This measurement 
and others are deliberate, using specifications that anticipate your paper 
as one part of the entire proceedings, and not as an independent document. 
Please do not revise any of the current designations.

\section{Prepare Your Paper Before Styling}
Before you begin to format your paper, first write and save the content as a 
separate text file. Complete all content and organizational editing before 
formatting. Please note sections \ref{AA}--\ref{SCM} below for more information on 
proofreading, spelling and grammar.

Keep your text and graphic files separate until after the text has been 
formatted and styled. Do not number text heads---{\LaTeX} will do that 
for you.

\subsection{Abbreviations and Acronyms}\label{AA}
Define abbreviations and acronyms the first time they are used in the text, 
even after they have been defined in the abstract. Abbreviations such as 
IEEE, SI, MKS, CGS, ac, dc, and rms do not have to be defined. Do not use 
abbreviations in the title or heads unless they are unavoidable.

\subsection{Units}
\begin{itemize}
\item Use either SI (MKS) or CGS as primary units. (SI units are encouraged.) English units may be used as secondary units (in parentheses). An exception would be the use of English units as identifiers in trade, such as ``3.5-inch disk drive''.
\item Avoid combining SI and CGS units, such as current in amperes and magnetic field in oersteds. This often leads to confusion because equations do not balance dimensionally. If you must use mixed units, clearly state the units for each quantity that you use in an equation.
\item Do not mix complete spellings and abbreviations of units: ``Wb/m\textsuperscript{2}'' or ``webers per square meter'', not ``webers/m\textsuperscript{2}''. Spell out units when they appear in text: ``. . . a few henries'', not ``. . . a few H''.
\item Use a zero before decimal points: ``0.25'', not ``.25''. Use ``cm\textsuperscript{3}'', not ``cc''.)
\end{itemize}

\subsection{Equations}
Number equations consecutively. To make your 
equations more compact, you may use the solidus (~/~), the exp function, or 
appropriate exponents. Italicize Roman symbols for quantities and variables, 
but not Greek symbols. Use a long dash rather than a hyphen for a minus 
sign. Punctuate equations with commas or periods when they are part of a 
sentence, as in:
\begin{equation}
a+b=\gamma\label{eq}
\end{equation}

Be sure that the 
symbols in your equation have been defined before or immediately following 
the equation. Use ``\eqref{eq}'', not ``Eq.~\eqref{eq}'' or ``equation \eqref{eq}'', except at 
the beginning of a sentence: ``Equation \eqref{eq} is . . .''

\subsection{\LaTeX-Specific Advice}

Please use ``soft'' (e.g., \verb|\eqref{Eq}|) cross references instead
of ``hard'' references (e.g., \verb|(1)|). That will make it possible
to combine sections, add equations, or change the order of figures or
citations without having to go through the file line by line.

Please don't use the \verb|{eqnarray}| equation environment. Use
\verb|{align}| or \verb|{IEEEeqnarray}| instead. The \verb|{eqnarray}|
environment leaves unsightly spaces around relation symbols.

Please note that the \verb|{subequations}| environment in {\LaTeX}
will increment the main equation counter even when there are no
equation numbers displayed. If you forget that, you might write an
article in which the equation numbers skip from (17) to (20), causing
the copy editors to wonder if you've discovered a new method of
counting.

{\BibTeX} does not work by magic. It doesn't get the bibliographic
data from thin air but from .bib files. If you use {\BibTeX} to produce a
bibliography you must send the .bib files. 

{\LaTeX} can't read your mind. If you assign the same label to a
subsubsection and a table, you might find that Table I has been cross
referenced as Table IV-B3. 

{\LaTeX} does not have precognitive abilities. If you put a
\verb|\label| command before the command that updates the counter it's
supposed to be using, the label will pick up the last counter to be
cross referenced instead. In particular, a \verb|\label| command
should not go before the caption of a figure or a table.

Do not use \verb|\nonumber| inside the \verb|{array}| environment. It
will not stop equation numbers inside \verb|{array}| (there won't be
any anyway) and it might stop a wanted equation number in the
surrounding equation.

\subsection{Some Common Mistakes}\label{SCM}
\begin{itemize}
\item The word ``data'' is plural, not singular.
\item The subscript for the permeability of vacuum $\mu_{0}$, and other common scientific constants, is zero with subscript formatting, not a lowercase letter ``o''.
\item In American English, commas, semicolons, periods, question and exclamation marks are located within quotation marks only when a complete thought or name is cited, such as a title or full quotation. When quotation marks are used, instead of a bold or italic typeface, to highlight a word or phrase, punctuation should appear outside of the quotation marks. A parenthetical phrase or statement at the end of a sentence is punctuated outside of the closing parenthesis (like this). (A parenthetical sentence is punctuated within the parentheses.)
\item A graph within a graph is an ``inset'', not an ``insert''. The word alternatively is preferred to the word ``alternately'' (unless you really mean something that alternates).
\item Do not use the word ``essentially'' to mean ``approximately'' or ``effectively''.
\item In your paper title, if the words ``that uses'' can accurately replace the word ``using'', capitalize the ``u''; if not, keep using lower-cased.
\item Be aware of the different meanings of the homophones ``affect'' and ``effect'', ``complement'' and ``compliment'', ``discreet'' and ``discrete'', ``principal'' and ``principle''.
\item Do not confuse ``imply'' and ``infer''.
\item The prefix ``non'' is not a word; it should be joined to the word it modifies, usually without a hyphen.
\item There is no period after the ``et'' in the Latin abbreviation ``et al.''.
\item The abbreviation ``i.e.'' means ``that is'', and the abbreviation ``e.g.'' means ``for example''.
\end{itemize}
An excellent style manual for science writers is \cite{b7}.

\subsection{Authors and Affiliations}
\textbf{The class file is designed for, but not limited to, six authors.} A 
minimum of one author is required for all conference articles. Author names 
should be listed starting from left to right and then moving down to the 
next line. This is the author sequence that will be used in future citations 
and by indexing services. Names should not be listed in columns nor group by 
affiliation. Please keep your affiliations as succinct as possible (for 
example, do not differentiate among departments of the same organization).

\subsection{Identify the Headings}
Headings, or heads, are organizational devices that guide the reader through 
your paper. There are two types: component heads and text heads.

Component heads identify the different components of your paper and are not 
topically subordinate to each other. Examples include Acknowledgments and 
References and, for these, the correct style to use is ``Heading 5''. Use 
``figure caption'' for your Figure captions, and ``table head'' for your 
table title. Run-in heads, such as ``Abstract'', will require you to apply a 
style (in this case, italic) in addition to the style provided by the drop 
down menu to differentiate the head from the text.

Text heads organize the topics on a relational, hierarchical basis. For 
example, the paper title is the primary text head because all subsequent 
material relates and elaborates on this one topic. If there are two or more 
sub-topics, the next level head (uppercase Roman numerals) should be used 
and, conversely, if there are not at least two sub-topics, then no subheads 
should be introduced.

\subsection{Figures and Tables}
\paragraph{Positioning Figures and Tables} Place figures and tables at the top and 
bottom of columns. Avoid placing them in the middle of columns. Large 
figures and tables may span across both columns. Figure captions should be 
below the figures; table heads should appear above the tables. Insert 
figures and tables after they are cited in the text. Use the abbreviation 
``Fig.~\ref{fig}'', even at the beginning of a sentence.

\begin{table}[htbp]
\caption{Table Type Styles}
\begin{center}
\begin{tabular}{|c|c|c|c|}
\hline
\textbf{Table}&\multicolumn{3}{|c|}{\textbf{Table Column Head}} \\
\cline{2-4} 
\textbf{Head} & \textbf{\textit{Table column subhead}}& \textbf{\textit{Subhead}}& \textbf{\textit{Subhead}} \\
\hline
copy& More table copy$^{\mathrm{a}}$& &  \\
\hline
\multicolumn{4}{l}{$^{\mathrm{a}}$Sample of a Table footnote.}
\end{tabular}
\label{tab1}
\end{center}
\end{table}

\begin{figure}[htbp]
\centerline{\includegraphics{fig1.png}}
\caption{Example of a figure caption.}
\label{fig}
\end{figure}

Figure Labels: Use 8 point Times New Roman for Figure labels. Use words 
rather than symbols or abbreviations when writing Figure axis labels to 
avoid confusing the reader. As an example, write the quantity 
``Magnetization'', or ``Magnetization, M'', not just ``M''. If including 
units in the label, present them within parentheses. Do not label axes only 
with units. In the example, write ``Magnetization (A/m)'' or ``Magnetization 
\{A[m(1)]\}'', not just ``A/m''. Do not label axes with a ratio of 
quantities and units. For example, write ``Temperature (K)'', not 
``Temperature/K''.

\section*{Acknowledgment}

The preferred spelling of the word ``acknowledgment'' in America is without 
an ``e'' after the ``g''. Avoid the stilted expression ``one of us (R. B. 
G.) thanks $\ldots$''. Instead, try ``R. B. G. thanks$\ldots$''. Put sponsor 
acknowledgments in the unnumbered footnote on the first page.

\section*{References}

Please number citations consecutively within brackets \cite{b1}. The 
sentence punctuation follows the bracket \cite{b2}. Refer simply to the reference 
number, as in \cite{b3}---do not use ``Ref. \cite{b3}'' or ``reference \cite{b3}'' except at 
the beginning of a sentence: ``Reference \cite{b3} was the first $\ldots$''

Number footnotes separately in superscripts. Place the actual footnote at 
the bottom of the column in which it was cited. Do not put footnotes in the 
abstract or reference list. Use letters for table footnotes.

Unless there are six authors or more give all authors' names; do not use 
``et al.''. Papers that have not been published, even if they have been 
submitted for publication, should be cited as ``unpublished'' \cite{b4}. Papers 
that have been accepted for publication should be cited as ``in press'' \cite{b5}. 
Capitalize only the first word in a paper title, except for proper nouns and 
element symbols.

For papers published in translation journals, please give the English 
citation first, followed by the original foreign-language citation \cite{b6}.
\end{comment}
\begin{thebibliography}{00}
\bibitem{b1}  World Health Organization, “Billions of people still breathe unhealthy air: new WHO data,” WHO News, 04-Apr-2022. [Online]. Available: https://www.who.int/news/item/04-04-2022-billions-of-people-still-breathe-unhealthy-air-new-who-data  (Accessed: 01-Oct-2025).
\begin{comment}
    \bibitem{b2} J. Clerk Maxwell, A Treatise on Electricity and Magnetism, 3rd ed., vol. 2. Oxford: Clarendon, 1892, pp.68--73.
\bibitem{b3} I. S. Jacobs and C. P. Bean, ``Fine particles, thin films and exchange anisotropy,'' in Magnetism, vol. III, G. T. Rado and H. Suhl, Eds. New York: Academic, 1963, pp. 271--350.
\bibitem{b4} K. Elissa, ``Title of paper if known,'' unpublished.
\bibitem{b5} R. Nicole, ``Title of paper with only first word capitalized,'' J. Name Stand. Abbrev., in press.
\bibitem{b6} Y. Yorozu, M. Hirano, K. Oka, and Y. Tagawa, ``Electron spectroscopy studies on magneto-optical media and plastic substrate interface,'' IEEE Transl. J. Magn. Japan, vol. 2, pp. 740--741, August 1987 [Digests 9th Annual Conf. Magnetics Japan, p. 301, 1982].
\bibitem{b7} M. Young, The Technical Writer's Handbook. Mill Valley, CA: University Science, 1989.
\end{comment}

\end{thebibliography}
\vspace{12pt}

\end{document}
